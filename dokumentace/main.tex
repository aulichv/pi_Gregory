\documentclass[11pt,a4paper,twoside,openright]{report}

\usepackage[top=25mm,bottom=25mm,right=25mm,left=30mm,head=12.5mm,foot=12.5mm]{geometry}
\let\openright=\cleardoublepage

\input{macros}

\def\NazevPrace{Určení hodnoty Ludolfova čísla pomocí řady, kterou navrhl James Gregory}
\def\Kruh{159}
\def\AutorPrace{Vladislav Aulich}
\def\DatumOdevzdani{2021}

% Vedoucí práce: Jméno a příjmení s~tituly
\def\Vedouci{Ing. Jan Kohout}

% Fakulta a obor
\def\StudijniProgram{Fyzikální a výpočetní chemie}
\def\Fakulta{Fakulta chemicko-inženýrská}

\begin{document}
	
	%%% Titulní strana práce a další povinné informační strany

%%% Titulní strana práce

\pagestyle{empty}
\pagenumbering{gobble}
\hypersetup{pageanchor=false}

\begin{center}

\vspace{\stretch{8}}

\includegraphics[width=.8\textwidth]{img/logo}

\vspace{\stretch{8}}

{\Huge\bfseries\NazevPrace}

\vspace{8mm}
\mdseries{Semestrální práce}

\vspace{\stretch{8}}
\large
\begin{tabular}{rl}
Autor: & \AutorPrace \\
\noalign{\vspace{2mm}}
Kruh: & \Kruh\\
\noalign{\vspace{2mm}}
Studijní obor: & \StudijniProgram\\
\noalign{\vspace{2mm}}
Fakulta: & \Fakulta\\
\noalign{\vspace{2mm}}
Akademický rok: & 2021/2022\\
\noalign{\vspace{2mm}}
Předmět: & Úvod do programování a algoritmů \\
\noalign{\vspace{2mm}}
Vedoucí práce: & \Vedouci \\
\end{tabular}

\vspace{20mm}
Praha, \DatumOdevzdani
\end{center}

\pagenumbering{arabic}
	
	% Obsah
	\setcounter{tocdepth}{2}
	\tableofcontents
	
	\chapter{Teoretická část}
	\pagestyle{fancy}
	
	\section{Úvod}
	
	
	
	\section{Zadání úlohy}
	Určete hodnotu Ludolfova čísla pomocí řady, kterou navrhl James Gregory.
	
	\chapter{Implementace}
	\section{Postup řešení}
	\section{Zdrojový kód}
\begin{verbatim}
using System;
using System.Collections.Generic;
using System.Linq;
using System.Text;
using System.Threading.Tasks;

namespace semestralka_konsole
{
    class Program
    {
        /// <summary>
        /// Metoda, která pomocí řady Jamese Gregoryho
        /// stanový hodnotu Ludolfova čísla.
        /// </summary>
        /// <param name="n">Počet iterací (členů řady)</param>
        /// <returns>Hodnota pi datového typu decimal</returns>
        public static decimal VypocitejPi(uint n)
        {
            decimal soucet = 0;
            decimal pomocna = 0;
            for (int i = 0; i < n; i++)
            {

                pomocna = 4m / (1 + (2 * i));
                //Pro sudé koeficienty
                if (i % 2 == 0)
                    soucet += pomocna;
                //Pro liché koeficienty
                else
                    soucet -= pomocna;
            }
            return soucet;
        }
        static void Main(string[] args)
        {
            bool pokracovat = true;
            uint n = 0;
            do
            {
                Console.WriteLine("Zadejte počet iterací:");
                try
                {
                    n = uint.Parse(Console.ReadLine());
                    pokracovat = false;
                }
                catch (Exception e)
                {
                    Console.WriteLine("Zadejte prosím celé kladné číslo menší než  4 294 967 295");

                }
            }
            while (pokracovat);
            //Provedení funkce pro uživatelem zadaný počet iterací
            Console.WriteLine("Hodnota Ludolfova čísla pro {0} iterací je {1}", n, VypocitejPi(n));
            Console.ReadKey();
        }
    }
}
\end{verbatim}
	\section{Ukázka programu}
	\chapter*{Závěr}
	\pagestyle{empty}
	\addcontentsline{toc}{chapter}{Závěr}
	
	
	Závěr obsahuje shrnutí práce a vyjadřuje se k~míře splnění jejího zadání. Dále by se zde mělo objevit sebehodnocení studenta a informace o~tom, co nového se naučil a jak vnímal svou práci na projektu.
	
	
	%%% Seznam použité literatury
	\nocite{dokumentace}
	\nocite{medium}
	\nocite{citace}
	\nocite{Historie_pi}
	\nocite{wiki:pi}
	\nocite{zvyrazneni_kodu}
	\printbibliography[title={Seznam použité literatury},heading={bibintoc}]
	
	%%% Seznam obrázků
	\openright
	\listoffigures
	\addcontentsline{toc}{chapter}{Seznam obrázků}
	
	%%% Seznam tabulek
	\clearpage
	\listoftables
	\addcontentsline{toc}{chapter}{Seznam tabulek}
	
	%%% Přílohy k práci, existují-li. Každá příloha musí být alespoň jednou
	%%% odkazována z vlastního textu práce. Přílohy se číslují.
	
	%\part*{Přílohy}
	%\appendix
	
\end{document}
