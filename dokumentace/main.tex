\documentclass[11pt,a4paper,twoside,openright]{report}

\usepackage[top=25mm,bottom=25mm,right=25mm,left=30mm,head=12.5mm,foot=12.5mm]{geometry}
\let\openright=\cleardoublepage

\input{macros}

\def\NazevPrace{Určení hodnoty Ludolfova čísla pomocí řady, kterou navrhl James Gregory}
\def\Kruh{159}
\def\AutorPrace{Vladislav Aulich}
\def\DatumOdevzdani{2021}

% Vedoucí práce: Jméno a příjmení s~tituly
\def\Vedouci{Ing. Jan Kohout}

% Fakulta a obor
\def\StudijniProgram{Fyzikální a výpočetní chemie}
\def\Fakulta{Fakulta chemicko-inženýrská}

\begin{document}
	
	%%% Titulní strana práce a další povinné informační strany

%%% Titulní strana práce

\pagestyle{empty}
\pagenumbering{gobble}
\hypersetup{pageanchor=false}

\begin{center}

\vspace{\stretch{8}}

\includegraphics[width=.8\textwidth]{img/logo}

\vspace{\stretch{8}}

{\Huge\bfseries\NazevPrace}

\vspace{8mm}
\mdseries{Semestrální práce}

\vspace{\stretch{8}}
\large
\begin{tabular}{rl}
Autor: & \AutorPrace \\
\noalign{\vspace{2mm}}
Kruh: & \Kruh\\
\noalign{\vspace{2mm}}
Studijní obor: & \StudijniProgram\\
\noalign{\vspace{2mm}}
Fakulta: & \Fakulta\\
\noalign{\vspace{2mm}}
Akademický rok: & 2021/2022\\
\noalign{\vspace{2mm}}
Předmět: & Úvod do programování a algoritmů \\
\noalign{\vspace{2mm}}
Vedoucí práce: & \Vedouci \\
\end{tabular}

\vspace{20mm}
Praha, \DatumOdevzdani
\end{center}

\pagenumbering{arabic}
	
	% Obsah
	\setcounter{tocdepth}{2}
	\tableofcontents
	
	\chapter{Teoretická část}
	\pagestyle{fancy}
	
	\section{Úvod}
	Ludolfovo číslo patří do množiny iracionálních čísel, nelze jej tedy vyjádřit jako podíl dvou celých čísel. 
	
	
	\section{Zadání úlohy}
	Určete hodnotu Ludolfova čísla pomocí řady, kterou navrhl James Gregory.
	
	\chapter{Implementace}
	\section{Postup řešení}
	\section{Zdrojový kód}
        \begin{lstlisting}[caption=Výsledný zdrojový kód]
        using System;
        using System.Collections.Generic;
        using System.Linq;
        using System.Text;
        using System.Threading.Tasks;
        
        namespace semestralka_konsole
        {
            class Program
            {
                /// <summary>
                /// Metoda, která pomocí řady Jamese Gregoryho
                /// stanový hodnotu Ludolfova čísla.
                /// </summary>
                /// <param name="n">Počet iterací (členů řady)</param>
                /// <returns>Hodnota pi datového typu decimal</returns>
                public static decimal VypocitejPi(uint n)
                {
                    decimal soucet = 0;
                    decimal pomocna = 0;
                    for (int i = 0; i < n; i++)
                    {
        
                        pomocna = 4m / (1 + (2 * i));
                        //Pro sudé koeficienty
                        if (i % 2 == 0)
                            soucet += pomocna;
                        //Pro liché koeficienty
                        else
                            soucet -= pomocna;
                    }
                    return soucet;
                }
                static void Main(string[] args)
                {
                    bool pokracovat = true;
                    uint n = 0;
                    do
                    {
                        Console.WriteLine("Zadejte počet iterací:");
                        try
                        {
                            n = uint.Parse(Console.ReadLine());
                            pokracovat = false;
                        }
                        catch (Exception e)
                        {
                            Console.WriteLine("Zadejte prosím celé kladné číslo menší než  4 294 967 295");
        
                        }
                    }
                    while (pokracovat);
                    //Provedení funkce pro uživatelem zadaný počet iterací
                    Console.WriteLine("Hodnota Ludolfova čísla pro {0} iterací je {1}", n, VypocitejPi(n));
                    Console.ReadKey();
                }
            }
        }
        \end{lstlisting}
	\section{Ukázka programu}
	\chapter*{Závěr}
	\pagestyle{empty}
	\addcontentsline{toc}{chapter}{Závěr}
	Při práci na projektu jsem se dozvěděl mnoho zajímavostí o historii čísla $\pi$. Zejména o jeho významu v matematice a snahách o stanovení jeho hodnoty. 
	
	Dále jsem rozšířil své znalosti o sázení zdrojového kódu v sázecím jazyce \LaTeX. Původně jsem kód vykresloval pomocí balíčku verbatim, u toho se mi však nepovedlo nastvit kód podle mých představ. Nakonec jsem tedy použil balíček \uv{listings}. U něj jsem narážel na problémy s kódem obsahujícím diakritiku.
	
	K tvorbě dokumentace jsem využíval webový nástroj Overleaf a jednotlivé verze jsem verzoval v repozitáři umístěném na serveru Github. Nové zkušenosti jsem získal i přizpůsobením si vývojového prostředí Visual Studio.
	
	Práci na projektu považuji za přínosnou, domnívám se, že jsem zadání splnil.

	%%% Seznam použité literatury
	\nocite{dokumentace}
	\nocite{medium}
	\nocite{citace}
	\nocite{Historie_pi}
	\nocite{wiki:pi}
	\nocite{zvyrazneni_kodu}
	\nocite{clisting}
	\printbibliography[title={Seznam použité literatury},heading={bibintoc}]
	
	%%% Seznam obrázků
	\openright
	\listoffigures
	\addcontentsline{toc}{chapter}{Seznam obrázků}
	
	%%% Seznam tabulek
	\clearpage
	\listoftables
	\lstlistoflistings
	\addcontentsline{toc}{chapter}{Seznam tabulek}
	
	%%% Přílohy k práci, existují-li. Každá příloha musí být alespoň jednou
	%%% odkazována z vlastního textu práce. Přílohy se číslují.
	
	%\part*{Přílohy}
	%\appendix
	
\end{document}
